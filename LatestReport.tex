%%%%%%%%%%%%%%%%%%%%%%%%%%%%%%%%%%%%%%%%%%%%%%%%%%%%%%%%%%
%                                                                                      %
%         Bristol Project LaTex Template            %
%                                                                                      %
%%%%%%%%%%%%%%%%%%%%%%%%%%%%%%%%%%%%%%%%%%%%%%%%%%%%%%%%%%
%
%   Author: Alex Charles           Email: aep.charles@gmail.com
%
% -----------------------------------------------------------------------------------
%      PACKAGES & OTHER DOCUMENT CONFIGURATIONS
% -----------------------------------------------------------------------------------
\documentclass{article}
\usepackage{sectsty}
\usepackage[english]{babel}
% Font Package for XeLatex
\usepackage{fontspec}
\setmainfont{Avenir-Light}

\usepackage{fancyhdr}
\usepackage{lastpage}
\usepackage{extramarks}
\usepackage{gensymb}
\usepackage{amsmath}
\usepackage{lipsum}
\usepackage{float}
\usepackage{graphicx}
\graphicspath{{TempImg/}{Img/}}%<<<<<<<<< Location of Template Images and Other Images, Add folders here
\usepackage{subfig}
\usepackage{wrapfig}
\usepackage[font ={small,it}]{caption}
\usepackage{amsmath,amsfonts,amsthm} % Math packages
\usepackage{cite}
\usepackage{geometry}

% -----------------------------------------------------------------------------------
%                   NAMES & CLASS DEFINITION %<<<<<<<<< INSERT DETAILS HERE
% -----------------------------------------------------------------------------------
\newcommand{\AssignmentTitle}{Assignment Title Goes Here}
\newcommand{\ModuleTitle}{Title of Module Goes Here}
\newcommand{\University}{University of Bristol}
\newcommand{\Faculty}{Faculty of Engineering}
\newcommand{\UniCrest}{crestbris.png}
\newcommand{\UniLogo}{logobris.png}%<<<<<<<<< Make Sure Files are in the Template
%\newcommand{\GroupName}{Group 2}
\newcommand{\StudentNameA}{Alex Charles}
\newcommand{\StudentNumberA}{1301866}
\newcommand{\SupervisorNameA}{Supervisor}
\newcommand{\SupervisorEmailA}{supervisor@bristol.ac.uk}


% -----------------------------------------------------------------------------------
%                   WORD COUTNER - for XeLaTex
% -----------------------------------------------------------------------------------
\usepackage{xesearch}
\newcounter{words}
\newenvironment{counted}{%
  \setcounter{words}{0}
  \SearchList!{wordcount}{\stepcounter{words}}
    {a?,b?,c?,d?,e?,f?,g?,h?,i?,j?,k?,l?,m?,
    n?,o?,p?,q?,r?,s?,t?,u?,v?,w?,x?,y?,z?}
  \UndoBoundary{'}
  \SearchOrder{p;}}{%
  \StopSearching}


% -----------------------------------------------------------------------------------
%                   MARGINS, HEADERS & FOOTERS
% -----------------------------------------------------------------------------------
 \geometry{
 left=20mm,
 right=20mm,
 top=25mm,
 bottom=25mm,
 }
\linespread{1.05}

\pagestyle{fancy}
\lhead{\includegraphics[width = 0.2\textwidth]{\UniLogo}}
\chead{\ModuleTitle}
% \rhead{}
\lfoot{\StudentNameA}
\cfoot{}
\rfoot{Page \thepage\ of \pageref{LastPage}}
\renewcommand\headrulewidth{0.4pt}
\renewcommand\footrulewidth{0.4pt}

\setlength\parindent{0pt}

\newcommand{\horrule}[1]{\rule{\linewidth}{#1}}

% -----------------------------------------------------------------------------------
%               DOCUMENT STRUCTURE COMMANDS
% -----------------------------------------------------------------------------------
% To sort out the formatting of header and footer when a page...
% ... split occurs "within" a problem environment.
\newcommand{\enterProblemHeader}[1]{
\nobreak\extramarks{#1 (Cont.)}\nobreak
\nobreak\extramarks{#1}{}\nobreak
}
% To sort out the formatting of header and footer when a page...
% ... split occur "between" problem environments.
\newcommand{\exitProblemHeader}[1]{
\nobreak\extramarks{#1 (Cont.)}\nobreak
\nobreak\extramarks{#1}{}\nobreak
}

% -----------------------------------------------------------------------------------
\begin{document}

%----------------------------------------------------------------------------------------
                                  %	TITLE PAGE FORMAT
%----------------------------------------------------------------------------------------
\begin{titlepage}

	\center % Center everything on the page
%----------------------------------------------------------------------------------------
%	HEADING SECTION
%----------------------------------------------------------------------------------------
		\usefont{OT1}{bch}{b}{n}
		\normalfont \normalsize \textsc{\University} \\ [10pt]
		\normalfont \normalsize \textsc{\Faculty} \\ [25pt]
%----------------------------------------------------------------------------------------
%	LOGO SECTION - Adds Univeristy Crest to the Report
%----------------------------------------------------------------------------------------
		\includegraphics[width = 0.2\textwidth]{\UniCrest}\\[0.5cm]
%----------------------------------------------------------------------------------------
%	HEADING SECTION
%----------------------------------------------------------------------------------------
		\normalfont \normalsize \textsc{\ModuleTitle} \\ [25pt]
%----------------------------------------------------------------------------------------
%	TITLE SECTION
%----------------------------------------------------------------------------------------
		\horrule{0.5pt} \\[0.4cm]
		\huge \textbf{\AssignmentTitle} \\
		\horrule{2pt} \\[0.5cm]
%----------------------------------------------------------------------------------------
%	HEADING SECTION
%----------------------------------------------------------------------------------------
%		\normalfont \normalsize \textsc{\GroupName} \\ [25pt]
%----------------------------------------------------------------------------------------
%	AUTHOR SECTION
%----------------------------------------------------------------------------------------
\begin{minipage}{0.4\textwidth}
\begin{flushleft} \large
\emph{Supervisors:}\\
% Change Name
\textbf{\SupervisorNameA}
\end{flushleft}
\end{minipage}
~
\begin{minipage}{0.4\textwidth}
\begin{flushright} \large
\emph{Email:} \\
\SupervisorEmailA

\end{flushright}
\end{minipage}\\[1cm]

\begin{minipage}{0.4\textwidth}
\begin{flushleft} \large
\emph{Names:}\\
	\textbf{\StudentNameA}
\end{flushleft}
\end{minipage}
~
\begin{minipage}{0.4\textwidth}
\begin{flushright} \large
\emph{Candidate Number:} \\
(\StudentNumberA)\\
\end{flushright}
\end{minipage}\\[2cm]

%----------------------------------------------------------------------------------------
%	DATE SECTION
%----------------------------------------------------------------------------------------
\textit{{\large \today}}\\[1cm] % Date, change the \today to a set date if you want to be precise
%----------------------------------------------------------------------------------------
\vfill % Fill the rest of the page with whitespace
\end{titlepage}


\newpage

% -----------------------------------------------------------------------------------
%                             	 ABSTRACT
% -----------------------------------------------------------------------------------
\begin{abstract}
\end{abstract}
% -----------------------------------------------------------------------------------
%                              TABLE OF CONTENTS
% -----------------------------------------------------------------------------------
\tableofcontents
\newpage
%% -----------------------------------------------------------------------------------
%%                          	  INTRODUCTION
%% -----------------------------------------------------------------------------------
\begin{counted} %<<<<<<<<<<<<<<STARTS WORD COUNTER
\documentclass[]{article}
\usepackage{lmodern}
\usepackage{amssymb,amsmath}
\usepackage{ifxetex,ifluatex}
\usepackage{fixltx2e} % provides \textsubscript
\ifnum 0\ifxetex 1\fi\ifluatex 1\fi=0 % if pdftex
  \usepackage[T1]{fontenc}
  \usepackage[utf8]{inputenc}
\else % if luatex or xelatex
  \ifxetex
    \usepackage{mathspec}
  \else
    \usepackage{fontspec}
  \fi
  \defaultfontfeatures{Ligatures=TeX,Scale=MatchLowercase}
\fi
% use upquote if available, for straight quotes in verbatim environments
\IfFileExists{upquote.sty}{\usepackage{upquote}}{}
% use microtype if available
\IfFileExists{microtype.sty}{%
\usepackage{microtype}
\UseMicrotypeSet[protrusion]{basicmath} % disable protrusion for tt fonts
}{}
\usepackage{hyperref}
\hypersetup{unicode=true,
            pdfborder={0 0 0},
            breaklinks=true}
\urlstyle{same}  % don't use monospace font for urls
\usepackage{color}
\usepackage{fancyvrb}
\newcommand{\VerbBar}{|}
\newcommand{\VERB}{\Verb[commandchars=\\\{\}]}
\DefineVerbatimEnvironment{Highlighting}{Verbatim}{commandchars=\\\{\}}
% Add ',fontsize=\small' for more characters per line
\newenvironment{Shaded}{}{}
\newcommand{\KeywordTok}[1]{\textcolor[rgb]{0.00,0.44,0.13}{\textbf{{#1}}}}
\newcommand{\DataTypeTok}[1]{\textcolor[rgb]{0.56,0.13,0.00}{{#1}}}
\newcommand{\DecValTok}[1]{\textcolor[rgb]{0.25,0.63,0.44}{{#1}}}
\newcommand{\BaseNTok}[1]{\textcolor[rgb]{0.25,0.63,0.44}{{#1}}}
\newcommand{\FloatTok}[1]{\textcolor[rgb]{0.25,0.63,0.44}{{#1}}}
\newcommand{\ConstantTok}[1]{\textcolor[rgb]{0.53,0.00,0.00}{{#1}}}
\newcommand{\CharTok}[1]{\textcolor[rgb]{0.25,0.44,0.63}{{#1}}}
\newcommand{\SpecialCharTok}[1]{\textcolor[rgb]{0.25,0.44,0.63}{{#1}}}
\newcommand{\StringTok}[1]{\textcolor[rgb]{0.25,0.44,0.63}{{#1}}}
\newcommand{\VerbatimStringTok}[1]{\textcolor[rgb]{0.25,0.44,0.63}{{#1}}}
\newcommand{\SpecialStringTok}[1]{\textcolor[rgb]{0.73,0.40,0.53}{{#1}}}
\newcommand{\ImportTok}[1]{{#1}}
\newcommand{\CommentTok}[1]{\textcolor[rgb]{0.38,0.63,0.69}{\textit{{#1}}}}
\newcommand{\DocumentationTok}[1]{\textcolor[rgb]{0.73,0.13,0.13}{\textit{{#1}}}}
\newcommand{\AnnotationTok}[1]{\textcolor[rgb]{0.38,0.63,0.69}{\textbf{\textit{{#1}}}}}
\newcommand{\CommentVarTok}[1]{\textcolor[rgb]{0.38,0.63,0.69}{\textbf{\textit{{#1}}}}}
\newcommand{\OtherTok}[1]{\textcolor[rgb]{0.00,0.44,0.13}{{#1}}}
\newcommand{\FunctionTok}[1]{\textcolor[rgb]{0.02,0.16,0.49}{{#1}}}
\newcommand{\VariableTok}[1]{\textcolor[rgb]{0.10,0.09,0.49}{{#1}}}
\newcommand{\ControlFlowTok}[1]{\textcolor[rgb]{0.00,0.44,0.13}{\textbf{{#1}}}}
\newcommand{\OperatorTok}[1]{\textcolor[rgb]{0.40,0.40,0.40}{{#1}}}
\newcommand{\BuiltInTok}[1]{{#1}}
\newcommand{\ExtensionTok}[1]{{#1}}
\newcommand{\PreprocessorTok}[1]{\textcolor[rgb]{0.74,0.48,0.00}{{#1}}}
\newcommand{\AttributeTok}[1]{\textcolor[rgb]{0.49,0.56,0.16}{{#1}}}
\newcommand{\RegionMarkerTok}[1]{{#1}}
\newcommand{\InformationTok}[1]{\textcolor[rgb]{0.38,0.63,0.69}{\textbf{\textit{{#1}}}}}
\newcommand{\WarningTok}[1]{\textcolor[rgb]{0.38,0.63,0.69}{\textbf{\textit{{#1}}}}}
\newcommand{\AlertTok}[1]{\textcolor[rgb]{1.00,0.00,0.00}{\textbf{{#1}}}}
\newcommand{\ErrorTok}[1]{\textcolor[rgb]{1.00,0.00,0.00}{\textbf{{#1}}}}
\newcommand{\NormalTok}[1]{{#1}}
\usepackage{longtable,booktabs}
\IfFileExists{parskip.sty}{%
\usepackage{parskip}
}{% else
\setlength{\parindent}{0pt}
\setlength{\parskip}{6pt plus 2pt minus 1pt}
}
\setlength{\emergencystretch}{3em}  % prevent overfull lines
\providecommand{\tightlist}{%
  \setlength{\itemsep}{0pt}\setlength{\parskip}{0pt}}
\setcounter{secnumdepth}{0}
% Redefines (sub)paragraphs to behave more like sections
\ifx\paragraph\undefined\else
\let\oldparagraph\paragraph
\renewcommand{\paragraph}[1]{\oldparagraph{#1}\mbox{}}
\fi
\ifx\subparagraph\undefined\else
\let\oldsubparagraph\subparagraph
\renewcommand{\subparagraph}[1]{\oldsubparagraph{#1}\mbox{}}
\fi

\date{}

\begin{document}

\section{Welcome to Marxico}\label{welcome-to-marxico}

\textbf{Marxico} is a delicate Markdown editor for Evernote. With
reliable storage and sync powered by Evernote, \textbf{Marxico} offers
greate writing experience.

\begin{itemize}
\tightlist
\item
  \textbf{Versatile} - supporting code highlight, \emph{LaTeX} \& flow
  charts, inserting images \& attachments by all means.
\item
  \textbf{Exquisite} - neat but powerful editor, featuring offline docs,
  live preview, and offering the {[}desktop client{]}{[}1{]} and offline
  {[}Chrome App{]}{[}2{]}.
\item
  \textbf{Sophisticated} - deeply integrated with Evernote, supporting
  notebook \& tags, two-way bind editing.
\end{itemize}

\begin{center}\rule{0.5\linewidth}{\linethickness}\end{center}

\subsection{Introducing Markdown}\label{introducing-markdown}

\begin{quote}
Markdown is a plain text formatting syntax designed to be converted to
HTML. Markdown is popularly used as format for readme files, \ldots{} or
in text editors for the quick creation of rich text documents. -
\href{http://en.wikipedia.org/wiki/Markdown}{Wikipedia}
\end{quote}

As showed in this manual, it uses hash(\#) to identify headings,
emphasizes some text to be \textbf{bold} or \emph{italic}. You can
insert a \href{http://www.example.com}{link} , or a
footnote{[}\^{}demo{]}. Serveral advanced syntax are listed below,
please press \texttt{Cmd\ +\ /} to view Markdown cheatsheet.

\subsubsection{Code block}\label{code-block}

\begin{Shaded}
\begin{Highlighting}[]
\AttributeTok{@requires_authorization}
\KeywordTok{def} \NormalTok{somefunc(param1}\OperatorTok{=}\StringTok{''}\NormalTok{, param2}\OperatorTok{=}\DecValTok{0}\NormalTok{):}
    \CommentTok{'''A docstring'''}
    \ControlFlowTok{if} \NormalTok{param1 }\OperatorTok{>} \NormalTok{param2: }\CommentTok{# interesting}
        \BuiltInTok{print} \StringTok{'Greater'}
    \ControlFlowTok{return} \NormalTok{(param2 }\OperatorTok{-} \NormalTok{param1 }\OperatorTok{+} \DecValTok{1}\NormalTok{) }\OperatorTok{or} \VariableTok{None}
\KeywordTok{class} \NormalTok{SomeClass:}
    \ControlFlowTok{pass}
\OperatorTok{>>>} \NormalTok{message }\OperatorTok{=} \StringTok{'''interpreter}
\StringTok{... prompt'''}
\end{Highlighting}
\end{Shaded}

\subsubsection{LaTeX expression}\label{latex-expression}

\[  x = \dfrac{-b \pm \sqrt{b^2 - 4ac}}{2a} \]

\subsubsection{Table}\label{table}

\begin{longtable}[c]{@{}lrc@{}}
\toprule
Item & Value & Qty\tabularnewline
\midrule
\endhead
Computer & 1600 USD & 5\tabularnewline
Phone & 12 USD & 12\tabularnewline
Pipe & 1 USD & 234\tabularnewline
\bottomrule
\end{longtable}

\subsubsection{Diagrams}\label{diagrams}

\paragraph{Flow charts}\label{flow-charts}

\begin{verbatim}
st=>start: Start
e=>end
op=>operation: My Operation
cond=>condition: Yes or No?

st->op->cond
cond(yes)->e
cond(no)->op
\end{verbatim}

\paragraph{Sequence diagrams}\label{sequence-diagrams}

\begin{verbatim}
Alice->Bob: Hello Bob, how are you?
Note right of Bob: Bob thinks
Bob-->Alice: I am good thanks!
\end{verbatim}

\begin{quote}
\textbf{Note:} You can find more information:
\end{quote}

\begin{quote}
\begin{itemize}
\tightlist
\item
  about \textbf{Sequence diagrams} syntax {[}here{]}{[}3{]},
\item
  about \textbf{Flow charts} syntax {[}here{]}{[}4{]}.
\end{itemize}
\end{quote}

\subsubsection{Checkbox}\label{checkbox}

You can use \texttt{-\ {[}\ {]}} and \texttt{-\ {[}x{]}} to create
checkboxes, for example:

\begin{itemize}
\tightlist
\item
  {[}x{]} Item1
\item
  {[} {]} Item2
\item
  {[} {]} Item3
\end{itemize}

\begin{quote}
\textbf{Note:} Currently it is only partially supported. You can't
toggle checkboxes in Evernote. You can only modify the Markdown in
Marxico to do that. Next version will fix this.
\end{quote}

\subsubsection{Dancing with Evernote}\label{dancing-with-evernote}

\paragraph{Notebook \& Tags}\label{notebook-tags}

\textbf{Marxico} add
\texttt{@(Notebook){[}tag1\textbar{}tag2\textbar{}tag3{]}} syntax to
select notebook and set tags for the note. After typing \texttt{@(}, the
notebook list would appear, please select one from it.

\paragraph{Title}\label{title}

\textbf{Marxico} would adopt the first heading encountered as the note
title. For example, in this manual the first line
\texttt{Welcome\ to\ Marxico} is the title.

\paragraph{Quick Editing}\label{quick-editing}

Note saved by \textbf{Marxico} in Evernote would have a red ribbon
button on the top-right corner. Click it and it would bring you back to
\textbf{Marxico} to edit the note.

\begin{quote}
\textbf{Note:} Currently \textbf{Marxico} is unable to detect and merge
any modifications in Evernote by user. Please go back to
\textbf{Marxico} to edit.
\end{quote}

\paragraph{Data Synchronization}\label{data-synchronization}

While saving rich HTML content in Evernote, \textbf{Marxico} puts the
Markdown text in a hidden area of the note, which makes it possible to
get the original text in \textbf{Marxico} and edit it again. This is a
really brilliant design because:

\begin{itemize}
\tightlist
\item
  it is beyond just one-way exporting HTML which other services do;
\item
  and it avoids privacy and security problems caused by storing content
  in a intermediate server.
\end{itemize}

\begin{quote}
\textbf{Privacy Statement: All of your notes data are saved in Evernote.
Marxico doesn't save any of them.}
\end{quote}

\paragraph{Offline Storage}\label{offline-storage}

\textbf{Marxico} stores your unsynchronized content locally in browser
storage, so no worries about network and broswer crash. It also keeps
the recent file list you've edited in
\texttt{Document\ Management(Cmd\ +\ O)}.

\begin{quote}
\textbf{Note:} Opthough browser storage is reliable in the most time,
Evernote is born to do that. So please sync the document regularly while
writing.
\end{quote}

\subsection{Shortcuts}\label{shortcuts}

Help \texttt{Cmd\ +\ /} Sync Doc \texttt{Cmd\ +\ S} Create Doc
\texttt{Cmd\ +\ Opt\ +\ N} Maximize Editor \texttt{Cmd\ +\ Enter}
Preview Doc \texttt{Cmd\ +\ Opt\ +\ Enter} Doc Management
\texttt{Cmd\ +\ O} Menu \texttt{Cmd\ +\ M}

Bold \texttt{Cmd\ +\ B} Insert Image \texttt{Cmd\ +\ G} Insert Link
\texttt{Cmd\ +\ L} Convert Heading \texttt{Cmd\ +\ H}

\subsection{About Pro}\label{about-pro}

\textbf{Marixo} offers a free trial of 10 days. After that, you need to
\href{http://marxi.co/purchase.html}{purchase} the Pro service.
Otherwise, you would not be able to sync new notes. Previous notes can
be edited and synced all the time.

\subsection{Credits}\label{credits}

\textbf{Marxico} was first built upon {[}Dillinger{]}{[}5{]}, and the
newest version is almost based on the awesome {[}StackEdit{]}{[}6{]}.
Acknowledgments to them and other incredible open source projects!

\subsection{Feedback \& Bug Report}\label{feedback-bug-report}

\begin{itemize}
\tightlist
\item
  Twitter: {[}@gock2{]}{[}7{]}
\item
  Email:
  \href{mailto:hustgock@gmail.com}{\nolinkurl{hustgock@gmail.com}}
\end{itemize}

\begin{center}\rule{0.5\linewidth}{\linethickness}\end{center}

Thank you for reading this manual. Now please press \texttt{Cmd\ +\ M}
and click \texttt{Link\ with\ Evernote}. Enjoy your \textbf{Marxico}
journey!

\end{document}
% -----------------------------------------------------------------------------------
%                                  APENDIX
% -----------------------------------------------------------------------------------

\end{counted} %<<<<<<<<<<<<<<ENDS WORD COUNTER

\newpage
\section{Appendices}
Above were \thewords\ words. %<<<<<<<<<<<<<<DISPLAYS WORD COUNTER
% -----------------------------------------------------------------------------------
%                               BIBLIOGRAPHY - Insert Name of BIB File Here
% -----------------------------------------------------------------------------------
\newpage

\bibliographystyle{plain}
\bibliography{BibFile}
\nocite{*}

\end{document}
